
\makeglossaries



%Glossario

\newglossaryentry{benchmark}
{
    name={Benchmark},
    description={Análise comparativa de desempenho; Teste comparativo de desempenho},
    plural={Benchmarks}
}

\newglossaryentry{API}
{
    name={API},
    description={API é a abreviação para \textit{Application Programming Interface}, Aplicação de Interface de Programação em português. Serve para a comunicação, por meio de protocolos de programação, entre duas aplicações distintas},
    plural={APIs}
}

\newglossaryentry{URL}
{
    name={URL},
    description={URL é o endereço lógico compreensível por seres humanos de alguma host dentro de uma rede de computadores. O significado da sigla é Uniform Resource Locator},
    plural={URLs}
}

\newglossaryentry{bucket}
{
    name={URL},
    description={bucket é um termo utilizado para se referir a recipientes lógicos onde dados são armazenados},
    plural={buckets}
}

\newglossaryentry{token}
{
    name={token},
    description={código de verificação de segurança},
    plural={tokens}
}

\newglossaryentry{Query}
{
    name={Query},
    description={requisição feita para uma operação sobre um banco de dados},
    plural={Queries}
}

\newglossaryentry{InfluxDB}
{
    name={InfluxDB},
    description={Plataforma para aquisição, armazenamento, análise e processamento de dados de séries temporais}
}

\newglossaryentry{posix}
{
    name={POSIX},
    description={ POSIX (um acrônimo para: Portable Operating System Interface, que pode ser traduzido como Interface Portável entre Sistemas Operativos) é uma família de normas definidas pelo IEEE para a manutenção de compatibilidade entre sistemas operacionais (sistemas operativos em PT-PT), e designada formalmente por IEEE 1003. }
}

\newglossaryentry{mpiio}
{
    name={MPI-IO},
    description={ MPI-IO, também conhecido como MPIIO, significa "Interface de Passagem de Mensagens para Entrada/Saída". É uma biblioteca de E/S (entrada/saída) paralela e um protocolo de comunicação frequentemente usado em ambientes de computação de alto desempenho (HPC). O MPI-IO é uma extensão do padrão Interface de Passagem de Mensagens (MPI), que é um modelo de programação amplamente utilizado para projetar aplicativos paralelos e distribuídos. }
}

\newglossaryentry{hdf5}
{
    name={HDF5},
    description={ HDF5 significa "Hierarchical Data Format version 5" (Formato de Dados Hierárquicos versão 5). É um formato de arquivo e uma biblioteca de software usado para armazenar, gerenciar e organizar dados heterogêneos e complexos, muitas vezes em aplicações científicas e de engenharia. O HDF5 oferece uma maneira flexível e eficiente de armazenar uma variedade de tipos de dados, desde dados numéricos até metadados descritivos. }
}


% \newglossaryentry{}
% {
%     name={},
%     description={}
% }

\newglossaryentry{ior}
{
    name={IOR},
    description={Integrated Benchmark for I/O. Projetado para medir desempenho de operações de entrada e saída de dados em sistemas de armazenamento de supercomputadores}
}

\newglossaryentry{storage}
{
    name={Storage},
    description={Servidor de Armazenamento de Dados},
    plural={Storages},
}

\newglossaryentry{thread}
{
    name={thread},
    description={Tarefa desenvolvida por um processo no sistema operacional},
    plural={threads},
}



%Acronimos

\newacronym{smart}{SMART}{Self-Monitoring, Analysis and Reporting Technology}
\newacronym[see=cad]{hpc}{HPC}{High-Performance Computing}
\newacronym{mpi}{MPI}{Message Passing Interface}
\newacronym[see=mpi]{ac:mpiio}{MPIIO}{Message Passing Interface - Input/Output; \gls{mpiio}}
\newacronym{ac:ior}{IOR}{ \gls{ior}}
\newacronym[see=hpc]{cad}{CAD}{Computação de Alto Desempenho}

% \newglossaryentry{maths}
% {
%     name=mathematics,
%     description={Mathematics is what mathematicians do}
% }

% \newglossaryentry{latex}
% {
%     name=latex,
%     description={Is a markup language specially suited for 
% scientific documents}
% }

% \newacronym{gcd}{GCD}{Greatest Common Divisor}

% \newacronym{lcm}{LCM}{Least Common Multiple}



% \newglossaryentry{formula}
% {
%         name=formula,
%         description={A mathematical expression}
% }



\makeglossaries
