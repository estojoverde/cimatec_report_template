%%%%%%% DO NOT TOUCH IT! Or it might crash... dont come asking for support later ¯\_(ツ)_/¯. 
\documentclass[../../main.tex]{subfiles}
\graphicspath{ {\subfix{../../CTsettings/figures/}} {./figures/} }
%%%%%%% DO NOT TOUCH IT! Or it might crash... dont come asking for support later ¯\_(ツ)_/¯. 


\begin{document}
%%%%%%% DO NOT TOUCH IT! Or it might crash... dont come asking for support later ¯\_(ツ)_/¯. 
\onlyinsubfile{
    \renewcommand{\onlyinsubfile}[1]{}
    \renewcommand{\notinsubfile}[1]{#1}

    \renewcommand{\onlyonmainfile}[1]{}
    \renewcommand{\onlyonpartfile}[1]{#1}
    \renewcommand{\onlyonchapterfile}[1]{}
    \renewcommand{\onlyonsectionfile}[1]{}

    \gdef\CTMcalibrefontpath{../../CTsettings/calibrefontfiles/}
    \setcalibrefont{\CTMcalibrefontpath}
    
    \fancypagestyle{plain}{\pagestyle{CTstylewithpage}}
    \pagestyle{CTstylewithpage}
    }
%%%%%%% DO NOT TOUCH IT! Or it might crash... dont come asking for support later ¯\_(ツ)_/¯. 



\CTpart{Part Example}

\lipsum[8-12]
\onlyonpartfile{
}
\important[color=green,name=Marcelo]{Este texto está em destaque entre duas chamadas a lipsum, no início de uma parte, antes do primeiro subfile ser chamado}

\onlyonpartfile{
\important[color=blue,name=Marcelo]{Este texto está logo abaixo, mas ele só será exibido quando a parte, tão e somente a parte, for compilada. }
}

\begin{comment}
    Isso acontece porque foi usado o comando \onlyonpartfile{}. E esse trecho não será compilado de maneira alguma, pois está dentro do ambiente de comentário. 
\end{comment}

\lipsum[8-12]

\subfile{./Chapters/chapter_potato}





\end{document}